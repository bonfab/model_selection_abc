\documentclass[a4paper, 11pt]{article}

\usepackage{geometry}
\geometry{a4paper,left=30mm,right=30mm, top=35mm, bottom=30mm}

\usepackage[english]{babel}
\usepackage[utf8]{inputenc} 
\usepackage[T1]{fontenc}
\usepackage{amsmath}
\usepackage{amssymb}
\usepackage{fancyhdr}
\usepackage{graphicx}
\usepackage{hyperref}
\usepackage{csquotes}

\usepackage{lastpage} % Seitenzahlen

\pagestyle{fancy}
\usepackage{amsmath}
\usepackage{tabularx} %schöne tabellen
\parindent0pt %einrücken verhindern
\renewcommand{\familydefault}{\sfdefault} % use sans fonts


\usepackage{polynom}
\cfoot{\thepage  \ / \pageref{LastPage}}

\usepackage[
backend=biber,
%style=ieee,
citestyle=authoryear
]{biblatex}

% % % % % % % % % % % % % % % % % % % % % % % %
% % % % % % % % % % % % % % % % % % % % % % % %
\newcommand{\thesis}{Bachelor Thesis}
\newcommand{\topic}{Inferring the Population Quantity of Multilocus Genotype Data}
\newcommand{\modul}{Bachelor Thesis}
\newcommand{\supervisor}{Supervisor: Manfred Opper}
\newcommand{\datum}{\today}
% % % % % % % % % % % % % % % % % % % % % % % %
% % % % % % % % % % % % % % % % % % % % % % % %


\addbibresource{sources.bib}

\begin{document} 

%%% Kopfzeile linker Bereich
%      gerade Seite   ungerade Seite
\lhead[ \leftmark   ]{\textbf{\modul}}
%%% Kopfzeile mittlerer Bereich
%      gerade Seite   ungerade Seite
\chead[\rightmark   ]{\rightmark{}}
%%% Kopfzeile linker Bereich
%      gerade Seite             ungerade Seite
\rhead[\textbf{text}]{\supervisor}


	%-- Deckblatt --						      
  \title{\textbf{\thesis\\[0.5cm] \Large{\topic} \\[1.5cm]}
	\author{%\tutorium\\ \\
		Fabian Bergmann, 372918\\ \\ \\ \\ \\ \\ \\ \\ 
		Pages: \pageref{LastPage} \\ \\ \\ \\
		}
	\normalsize{\supervisor}} %Thema ändern	
	\date{Submission Date: \datum} %Datum ändern
	\maketitle
	\newpage
	
\renewcommand{\labelenumi}{\alph{enumi})}
\renewcommand{\labelenumii}{(\roman{enumii})}
\renewcommand{\labelenumiii}{\arabic{enumiii}.}
%\renewcommand{\labelenumii}{\textbf{-}}	
%-- Eigentlicher Text --

\section*{Generating Data}

\subsection*{Biological Background}

\subsubsection*{Key words}

\begin{itemize}
\item \textbf{Chromosome:} A DNA molecule that encodes genetic information.

\item \textbf{Gene:} A DNA (or RNA) sequence that specifies the structure of a particular functional molecule.

\item \textbf{Locus:} A particular position on the chromosome, like the position of a specific gene.

\item \textbf{Allele:} A variant form of a given gene. Different alleles can lead to distinct phenotypic traits.

\end{itemize}

\subsection*{Admixture}
The subsequent admixture model, follows a model proposed by \cite{pritchard2000inference}. 



\newpage
\section*{Aufgabe 1: Titel (Punkte: 7)}

\emph{Für den folgenden Abschnitt solltet Ihr den Latex-Code mit der Ausgabe vergleichen.}

Einen neuen Absatz beginnt ihr durch das einfügen einer Leerzeile.
Hier beginnt die neue Zeile:

Zeilenumbrüche in der .tex Datei werden 
ignoriert.

\subsection*{Aufgabe 1.1: Unterpunkt (Punkte: 1)}

Kommentarzeilen werden in Latex mit \% begonnen.
%Dies ist ein Kommentar

Ihr werdet in diesem Kurs viel mit mathematischen Ausdrücken arbeiten.
Daher folgen nun die wichtigste Umgebung: \texttt{align} (bitte betrachtet wieder den Latex-Code).

\begin{align}
	a_1 = b^2 +4 \\
	a_1 - 4 = b^2
\end{align}

Wollt ihr die Zeilen nicht durchnummeriert haben, so müsst ihr folgendes ändern:

\begin{align*}
	a_1 = b^2 +4 \\
	a_1 - 4 = b^2
\end{align*}

Um mehr als ein Zeichen hoch- oder tiefgestellt darzustellen, müssen diese in geschweifte Klammern geschrieben werden.

\begin{align}
	a_{i+10} = b^{2+j} +4 \\
	a_{i+10} - 4 = b^{2+j}
\end{align}

Jetzt müssen noch die Gleichheitszeichen untereinander gesetzt werden.

\begin{align}
	a_1 &= b^2 +4 \\
	a_1 - 4 &= b^2
\end{align}

Matrizen können wie folgt dargestellt werden:

\begin{align*}
\begin{pmatrix}
1 & 2 & 3 \\
4 & 5 & 6
\end{pmatrix} 
+ 
\begin{pmatrix}
a & b & c\\
d & e & f
\end{pmatrix} 
&= 
\begin{pmatrix}
-3 & 6 & 12 \\
4.3 & -1.2 & 9 
\end{pmatrix}
\\[5pt]
\begin{pmatrix}
a & b & c\\
d & e & f
\end{pmatrix} 
&= 
\begin{pmatrix}
-4 & 4 & 9 \\
0.3 & -6.2 & 3 
\end{pmatrix}
\\[5pt]
&\Rightarrow a = -4 \\[2pt]
&\Rightarrow b = 4 \\[2pt]
&\Rightarrow c = 9 \\[2pt]
&\Rightarrow d = 0.3 \\[2pt]
&\Rightarrow e = -6.2 \\[2pt]
&\Rightarrow f = 3 
\end{align*}

Im Folgenden findet ihr eine Liste der wichtigsten Symbole und Zeichen. Weitere Zeichen findet ihr unter
\url{http://tug.ctan.org/info/symbols/comprehensive/symbols-a4.pdf}.

\begin{align*}
	\{ ~~~ \} \\
	\cup \\
	\cap \\
	\setminus \\
	\subset \\
	\subsetneq \\
	\subseteq \\
	\supset \\
	\in \\
	\notin \\
	\mathbb{N}\\
	\mathbb{Z}\\
	\mathbb{Q}\\
	\mathbb{R}\\
	\mathbb{C}\\
	\neq \\
	\approx \\
	\le \\
	\leq \\
	\leqq \\
	\ge \\
	\geq \\
	\alpha\\
	\beta \\
	\sum_{i=1}^{n} = x_i \\
	\langle ~ \rangle \\
	\cdot,\cdots \\
	\Rightarrow \\
	\rightarrow \\
	\vec{v}\\
\end{align*}

\printbibliography

\end{document}
